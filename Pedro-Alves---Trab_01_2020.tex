% Options for packages loaded elsewhere
\PassOptionsToPackage{unicode}{hyperref}
\PassOptionsToPackage{hyphens}{url}
%
\documentclass[
]{article}
\usepackage{lmodern}
\usepackage{amssymb,amsmath}
\usepackage{ifxetex,ifluatex}
\ifnum 0\ifxetex 1\fi\ifluatex 1\fi=0 % if pdftex
  \usepackage[T1]{fontenc}
  \usepackage[utf8]{inputenc}
  \usepackage{textcomp} % provide euro and other symbols
\else % if luatex or xetex
  \usepackage{unicode-math}
  \defaultfontfeatures{Scale=MatchLowercase}
  \defaultfontfeatures[\rmfamily]{Ligatures=TeX,Scale=1}
\fi
% Use upquote if available, for straight quotes in verbatim environments
\IfFileExists{upquote.sty}{\usepackage{upquote}}{}
\IfFileExists{microtype.sty}{% use microtype if available
  \usepackage[]{microtype}
  \UseMicrotypeSet[protrusion]{basicmath} % disable protrusion for tt fonts
}{}
\makeatletter
\@ifundefined{KOMAClassName}{% if non-KOMA class
  \IfFileExists{parskip.sty}{%
    \usepackage{parskip}
  }{% else
    \setlength{\parindent}{0pt}
    \setlength{\parskip}{6pt plus 2pt minus 1pt}}
}{% if KOMA class
  \KOMAoptions{parskip=half}}
\makeatother
\usepackage{xcolor}
\IfFileExists{xurl.sty}{\usepackage{xurl}}{} % add URL line breaks if available
\IfFileExists{bookmark.sty}{\usepackage{bookmark}}{\usepackage{hyperref}}
\hypersetup{
  pdftitle={Estatística Computacional I},
  hidelinks,
  pdfcreator={LaTeX via pandoc}}
\urlstyle{same} % disable monospaced font for URLs
\usepackage[margin=1in]{geometry}
\usepackage{color}
\usepackage{fancyvrb}
\newcommand{\VerbBar}{|}
\newcommand{\VERB}{\Verb[commandchars=\\\{\}]}
\DefineVerbatimEnvironment{Highlighting}{Verbatim}{commandchars=\\\{\}}
% Add ',fontsize=\small' for more characters per line
\usepackage{framed}
\definecolor{shadecolor}{RGB}{248,248,248}
\newenvironment{Shaded}{\begin{snugshade}}{\end{snugshade}}
\newcommand{\AlertTok}[1]{\textcolor[rgb]{0.94,0.16,0.16}{#1}}
\newcommand{\AnnotationTok}[1]{\textcolor[rgb]{0.56,0.35,0.01}{\textbf{\textit{#1}}}}
\newcommand{\AttributeTok}[1]{\textcolor[rgb]{0.77,0.63,0.00}{#1}}
\newcommand{\BaseNTok}[1]{\textcolor[rgb]{0.00,0.00,0.81}{#1}}
\newcommand{\BuiltInTok}[1]{#1}
\newcommand{\CharTok}[1]{\textcolor[rgb]{0.31,0.60,0.02}{#1}}
\newcommand{\CommentTok}[1]{\textcolor[rgb]{0.56,0.35,0.01}{\textit{#1}}}
\newcommand{\CommentVarTok}[1]{\textcolor[rgb]{0.56,0.35,0.01}{\textbf{\textit{#1}}}}
\newcommand{\ConstantTok}[1]{\textcolor[rgb]{0.00,0.00,0.00}{#1}}
\newcommand{\ControlFlowTok}[1]{\textcolor[rgb]{0.13,0.29,0.53}{\textbf{#1}}}
\newcommand{\DataTypeTok}[1]{\textcolor[rgb]{0.13,0.29,0.53}{#1}}
\newcommand{\DecValTok}[1]{\textcolor[rgb]{0.00,0.00,0.81}{#1}}
\newcommand{\DocumentationTok}[1]{\textcolor[rgb]{0.56,0.35,0.01}{\textbf{\textit{#1}}}}
\newcommand{\ErrorTok}[1]{\textcolor[rgb]{0.64,0.00,0.00}{\textbf{#1}}}
\newcommand{\ExtensionTok}[1]{#1}
\newcommand{\FloatTok}[1]{\textcolor[rgb]{0.00,0.00,0.81}{#1}}
\newcommand{\FunctionTok}[1]{\textcolor[rgb]{0.00,0.00,0.00}{#1}}
\newcommand{\ImportTok}[1]{#1}
\newcommand{\InformationTok}[1]{\textcolor[rgb]{0.56,0.35,0.01}{\textbf{\textit{#1}}}}
\newcommand{\KeywordTok}[1]{\textcolor[rgb]{0.13,0.29,0.53}{\textbf{#1}}}
\newcommand{\NormalTok}[1]{#1}
\newcommand{\OperatorTok}[1]{\textcolor[rgb]{0.81,0.36,0.00}{\textbf{#1}}}
\newcommand{\OtherTok}[1]{\textcolor[rgb]{0.56,0.35,0.01}{#1}}
\newcommand{\PreprocessorTok}[1]{\textcolor[rgb]{0.56,0.35,0.01}{\textit{#1}}}
\newcommand{\RegionMarkerTok}[1]{#1}
\newcommand{\SpecialCharTok}[1]{\textcolor[rgb]{0.00,0.00,0.00}{#1}}
\newcommand{\SpecialStringTok}[1]{\textcolor[rgb]{0.31,0.60,0.02}{#1}}
\newcommand{\StringTok}[1]{\textcolor[rgb]{0.31,0.60,0.02}{#1}}
\newcommand{\VariableTok}[1]{\textcolor[rgb]{0.00,0.00,0.00}{#1}}
\newcommand{\VerbatimStringTok}[1]{\textcolor[rgb]{0.31,0.60,0.02}{#1}}
\newcommand{\WarningTok}[1]{\textcolor[rgb]{0.56,0.35,0.01}{\textbf{\textit{#1}}}}
\usepackage{longtable,booktabs}
% Correct order of tables after \paragraph or \subparagraph
\usepackage{etoolbox}
\makeatletter
\patchcmd\longtable{\par}{\if@noskipsec\mbox{}\fi\par}{}{}
\makeatother
% Allow footnotes in longtable head/foot
\IfFileExists{footnotehyper.sty}{\usepackage{footnotehyper}}{\usepackage{footnote}}
\makesavenoteenv{longtable}
\usepackage{graphicx,grffile}
\makeatletter
\def\maxwidth{\ifdim\Gin@nat@width>\linewidth\linewidth\else\Gin@nat@width\fi}
\def\maxheight{\ifdim\Gin@nat@height>\textheight\textheight\else\Gin@nat@height\fi}
\makeatother
% Scale images if necessary, so that they will not overflow the page
% margins by default, and it is still possible to overwrite the defaults
% using explicit options in \includegraphics[width, height, ...]{}
\setkeys{Gin}{width=\maxwidth,height=\maxheight,keepaspectratio}
% Set default figure placement to htbp
\makeatletter
\def\fps@figure{htbp}
\makeatother
\setlength{\emergencystretch}{3em} % prevent overfull lines
\providecommand{\tightlist}{%
  \setlength{\itemsep}{0pt}\setlength{\parskip}{0pt}}
\setcounter{secnumdepth}{-\maxdimen} % remove section numbering

\title{Estatística Computacional I}
\usepackage{etoolbox}
\makeatletter
\providecommand{\subtitle}[1]{% add subtitle to \maketitle
  \apptocmd{\@title}{\par {\large #1 \par}}{}{}
}
\makeatother
\subtitle{Trabalho 02 - (21/05/2020)}
\author{}
\date{\vspace{-2.5em}}

\begin{document}
\maketitle

Orientações:

\begin{enumerate}
\def\labelenumi{\arabic{enumi}.}
\tightlist
\item
  Responda as questões deixando o código R nos respectivos chunks
\item
  Mantenha os nomes de objetos solicitados na questão
\item
  Sempre deixe por último o objeto que responde a pergunta
\item
  Salve o arquivo com o seu nome
\item
  Execute o Preview para gerar o html
\item
  Envie o html e Rmd pelo CR 7 . link do arquivo:
  {[}\url{https://bit.ly/2WRHoXB}{]}
\end{enumerate}

\hypertarget{leia-o-arquivo-forbes2000.csv-carregando-o-em-um-objeto-com-nome-df.}{%
\subsection{1- Leia o arquivo forbes2000.csv carregando-o em um objeto
com nome
df.}\label{leia-o-arquivo-forbes2000.csv-carregando-o-em-um-objeto-com-nome-df.}}

\begin{Shaded}
\begin{Highlighting}[]
\NormalTok{df =}\StringTok{ }\KeywordTok{read.csv}\NormalTok{(}\StringTok{"F:/files/R/Trabalho01/forbes2000.csv"}\NormalTok{, }\DataTypeTok{sep =} \StringTok{';'}\NormalTok{, }\DataTypeTok{dec =} \StringTok{','}\NormalTok{)}
\end{Highlighting}
\end{Shaded}

A lista Forbes 2000 é um ranking das maiores empresas do mundo, medido
por vendas, lucros, ativos e valor de mercado.

\hypertarget{renomeie-as-colunas-com-os-nomes-a-seguir-que-juxe1-estuxe3o-na-ordem-das-colunas-como-segue-rank-empresa-pais-categoria-vendas-lucro-ativo-valor.mantenha-o-nome-df.}{%
\subsection{2- Renomeie as colunas com os nomes a seguir, que já estão
na ordem das colunas como segue: rank, empresa, pais, categoria, vendas,
lucro, ativo, valor).Mantenha o nome
df.}\label{renomeie-as-colunas-com-os-nomes-a-seguir-que-juxe1-estuxe3o-na-ordem-das-colunas-como-segue-rank-empresa-pais-categoria-vendas-lucro-ativo-valor.mantenha-o-nome-df.}}

\begin{Shaded}
\begin{Highlighting}[]
\KeywordTok{colnames}\NormalTok{(df) =}\StringTok{ }\KeywordTok{c}\NormalTok{(}\StringTok{'rank'}\NormalTok{,}\StringTok{'empresa'}\NormalTok{,}\StringTok{'pais'}\NormalTok{,}\StringTok{'categoria'}\NormalTok{,}\StringTok{'vendas'}\NormalTok{,}\StringTok{'lucro'}\NormalTok{,}\StringTok{'ativo'}\NormalTok{,}\StringTok{'valor'}\NormalTok{)}
\end{Highlighting}
\end{Shaded}

\hypertarget{quantos-e-quais-pauxedses-estuxe3o-nesta-lista}{%
\subsection{3- Quantos e quais países estão nesta
lista?}\label{quantos-e-quais-pauxedses-estuxe3o-nesta-lista}}

\begin{Shaded}
\begin{Highlighting}[]
\NormalTok{n_paises =}\StringTok{ }\KeywordTok{length}\NormalTok{(}\KeywordTok{unique}\NormalTok{(df}\OperatorTok{$}\NormalTok{pais))}
\NormalTok{paises =}\StringTok{ }\KeywordTok{unique}\NormalTok{(df}\OperatorTok{$}\NormalTok{pais)}
\end{Highlighting}
\end{Shaded}

\hypertarget{qual-suxe3o-os-10-paises-que-aparecem-com-mais-frequencia-fauxe7a-uma-tabela-com-o-nome-do-pauxeds-e-a-frequencia.}{%
\subsection{4- Qual são os 10 paises que aparecem com mais frequencia?
Faça uma tabela com o nome do país e a
frequencia.}\label{qual-suxe3o-os-10-paises-que-aparecem-com-mais-frequencia-fauxe7a-uma-tabela-com-o-nome-do-pauxeds-e-a-frequencia.}}

\begin{Shaded}
\begin{Highlighting}[]
\NormalTok{freq_paises =}\StringTok{ }\KeywordTok{sort}\NormalTok{(}\KeywordTok{table}\NormalTok{(df}\OperatorTok{$}\NormalTok{pais),}\DataTypeTok{decreasing =} \OtherTok{TRUE}\NormalTok{)}
\NormalTok{freq_paises =}\StringTok{ }\NormalTok{freq_paises[}\DecValTok{1}\OperatorTok{:}\DecValTok{10}\NormalTok{]}
\KeywordTok{library}\NormalTok{(knitr)}
\KeywordTok{kable}\NormalTok{(freq_paises, }\DataTypeTok{row.names =} \OtherTok{FALSE}\NormalTok{, }\DataTypeTok{col.names =} \KeywordTok{c}\NormalTok{(}\StringTok{'Países'}\NormalTok{,}\StringTok{'Frequência'))}
\end{Highlighting}
\end{Shaded}

\begin{longtable}[]{@{}lr@{}}
\toprule
Países & Frequência\tabularnewline
\midrule
\endhead
United States & 751\tabularnewline
Japan & 316\tabularnewline
United Kingdom & 137\tabularnewline
Germany & 65\tabularnewline
France & 63\tabularnewline
Canada & 56\tabularnewline
South Korea & 45\tabularnewline
Italy & 41\tabularnewline
Australia & 37\tabularnewline
Taiwan & 35\tabularnewline
\bottomrule
\end{longtable}

\hypertarget{quantas-e-quais-suxe3o-as-empresas-brasileiras-a-lista-de-empresas-deve-estar-em-ordem-alfabuxe9tica}{%
\subsection{5- Quantas e quais são as empresas brasileiras? A lista de
empresas deve estar em ordem
alfabética}\label{quantas-e-quais-suxe3o-as-empresas-brasileiras-a-lista-de-empresas-deve-estar-em-ordem-alfabuxe9tica}}

\begin{Shaded}
\begin{Highlighting}[]
\NormalTok{empresas_br =}\StringTok{ }\KeywordTok{sort}\NormalTok{(df}\OperatorTok{$}\NormalTok{empresa[}\KeywordTok{which}\NormalTok{(df}\OperatorTok{$}\NormalTok{pais }\OperatorTok{==}\StringTok{ 'Brazil'}\NormalTok{)])}
\KeywordTok{kable}\NormalTok{(empresas_br,}\DataTypeTok{row.names =} \OtherTok{FALSE}\NormalTok{, }\DataTypeTok{col.names =} \StringTok{"Empresas Brasileiras"}\NormalTok{)}
\end{Highlighting}
\end{Shaded}

\begin{longtable}[]{@{}l@{}}
\toprule
Empresas Brasileiras\tabularnewline
\midrule
\endhead
AmBev\tabularnewline
Aracruz Celulose\tabularnewline
Banco Bradesco Group\tabularnewline
Banco do Brasil\tabularnewline
Brasil Telecom\tabularnewline
CBD-Brasil Distribuieco\tabularnewline
CSN-Cia Siderurgica\tabularnewline
Eletrobras\tabularnewline
Embraer\tabularnewline
Itazsa\tabularnewline
Metalurgica Gerdau\tabularnewline
Petrobras-Petrsleo Brasil\tabularnewline
Tele Norte Leste\tabularnewline
Unibanco Group\tabularnewline
Vale do Rio Doce\tabularnewline
\bottomrule
\end{longtable}

\hypertarget{compare-os-valor-muxe9dio-das-empresas-brasileiras-com-as-demais-empresas.-calcule-a-muxe9dia-e-variuxe2ncia.-o-resultado-deve-ser-uma-tabela-com-as-colunas-pauxedsmuxe9dia_vendas-e-variuxe2ncia_vendas.}{%
\subsection{6- Compare os valor médio das empresas brasileiras com as
demais empresas. Calcule a média e variância. O resultado deve ser uma
tabela com as colunas País,Média\_Vendas e
Variância\_Vendas.}\label{compare-os-valor-muxe9dio-das-empresas-brasileiras-com-as-demais-empresas.-calcule-a-muxe9dia-e-variuxe2ncia.-o-resultado-deve-ser-uma-tabela-com-as-colunas-pauxedsmuxe9dia_vendas-e-variuxe2ncia_vendas.}}

\begin{Shaded}
\begin{Highlighting}[]
\NormalTok{brasil_dados =}\StringTok{ }\KeywordTok{c}\NormalTok{(}\StringTok{"Brasil"}\NormalTok{,}\KeywordTok{mean}\NormalTok{(df}\OperatorTok{$}\NormalTok{valor[}\KeywordTok{which}\NormalTok{(df}\OperatorTok{$}\NormalTok{pais }\OperatorTok{==}\StringTok{ 'Brazil'}\NormalTok{)]),}\KeywordTok{var}\NormalTok{(df}\OperatorTok{$}\NormalTok{valor[}\KeywordTok{which}\NormalTok{(df}\OperatorTok{$}\NormalTok{pais }\OperatorTok{==}\StringTok{ 'Brazil'}\NormalTok{)]))}
\NormalTok{mundo_dados =}\StringTok{ }\KeywordTok{c}\NormalTok{(}\StringTok{"Restante do Mundo"}\NormalTok{,}\KeywordTok{mean}\NormalTok{(df}\OperatorTok{$}\NormalTok{valor[}\KeywordTok{which}\NormalTok{(df}\OperatorTok{$}\NormalTok{pais }\OperatorTok{!=}\StringTok{ 'Brazil'}\NormalTok{)]),}\KeywordTok{var}\NormalTok{(df}\OperatorTok{$}\NormalTok{valor[}\KeywordTok{which}\NormalTok{(df}\OperatorTok{$}\NormalTok{pais }\OperatorTok{!=}\StringTok{ 'Brazil'}\NormalTok{)]))}
\KeywordTok{kable}\NormalTok{(}\KeywordTok{matrix}\NormalTok{(}\KeywordTok{c}\NormalTok{(brasil_dados,mundo_dados),}\DataTypeTok{nrow =} \DecValTok{2}\NormalTok{,}\DataTypeTok{ncol =} \DecValTok{3}\NormalTok{,}\DataTypeTok{byrow =} \OtherTok{TRUE}\NormalTok{), }\DataTypeTok{row.names =} \OtherTok{FALSE}\NormalTok{, }\DataTypeTok{col.names =} \KeywordTok{c}\NormalTok{(}\StringTok{"País"}\NormalTok{, }\StringTok{"Média_Valor"}\NormalTok{, }\StringTok{"Variância_Valor"}\NormalTok{))}
\end{Highlighting}
\end{Shaded}

\begin{longtable}[]{@{}lll@{}}
\toprule
País & Média\_Valor & Variância\_Valor\tabularnewline
\midrule
\endhead
Brasil & 7.718 & 81.5038742857143\tabularnewline
Restante do Mundo & 11.9090881612091 & 602.117730468534\tabularnewline
\bottomrule
\end{longtable}

\begin{Shaded}
\begin{Highlighting}[]
\NormalTok{Media_vendas =}\DecValTok{1}\OperatorTok{:}\KeywordTok{length}\NormalTok{(paises)}
\ControlFlowTok{for}\NormalTok{ (i }\ControlFlowTok{in} \DecValTok{1}\OperatorTok{:}\KeywordTok{length}\NormalTok{(paises)) \{}
\NormalTok{Media_vendas[i] =}\StringTok{ }\KeywordTok{mean}\NormalTok{(df}\OperatorTok{$}\NormalTok{vendas[}\KeywordTok{which}\NormalTok{(df}\OperatorTok{$}\NormalTok{pais }\OperatorTok{==}\StringTok{ }\NormalTok{paises[i])])}
\NormalTok{\}}
\NormalTok{variancia_vendas =}\DecValTok{1}\OperatorTok{:}\KeywordTok{length}\NormalTok{(paises)}
\ControlFlowTok{for}\NormalTok{ (i }\ControlFlowTok{in} \DecValTok{1}\OperatorTok{:}\KeywordTok{length}\NormalTok{(paises)) \{}
\NormalTok{variancia_vendas[i] =}\StringTok{ }\KeywordTok{var}\NormalTok{(df}\OperatorTok{$}\NormalTok{vendas[}\KeywordTok{which}\NormalTok{(df}\OperatorTok{$}\NormalTok{pais }\OperatorTok{==}\StringTok{ }\NormalTok{paises[i])])}
\NormalTok{\}}

\KeywordTok{kable}\NormalTok{(}\KeywordTok{matrix}\NormalTok{(}\KeywordTok{c}\NormalTok{(paises,Media_vendas,variancia_vendas),}\DataTypeTok{nrow =} \DecValTok{61}\NormalTok{,}\DataTypeTok{ncol =} \DecValTok{3}\NormalTok{,}\DataTypeTok{byrow =} \OtherTok{FALSE}\NormalTok{), }\DataTypeTok{row.names =} \OtherTok{FALSE}\NormalTok{, }\DataTypeTok{col.names =} \KeywordTok{c}\NormalTok{(}\StringTok{"País"}\NormalTok{, }\StringTok{"Média_Vendas"}\NormalTok{, }\StringTok{"Variância_Vendas"}\NormalTok{))}
\end{Highlighting}
\end{Shaded}

\begin{longtable}[]{@{}lll@{}}
\toprule
País & Média\_Vendas & Variância\_Vendas\tabularnewline
\midrule
\endhead
United States & 10.0582556591212 & 398.913099753218\tabularnewline
United Kingdom & 10.4451094890511 & 474.889469289395\tabularnewline
Japan & 10.1906329113924 & 303.201913883866\tabularnewline
Switzerland & 12.4567647058824 & 261.801513458111\tabularnewline
Netherlands & 17.0207142857143 & 481.58776984127\tabularnewline
Netherlands/ United Kingdom & 92.1 & 3427.92\tabularnewline
France & 20.1020634920635 & 610.974752124936\tabularnewline
Germany & 20.7813846153846 & 803.000330865385\tabularnewline
Italy & 10.2139024390244 & 235.891484390244\tabularnewline
Spain & 7.84344827586207 & 84.4742662561576\tabularnewline
South Korea & 7.96933333333333 & 107.862192727273\tabularnewline
China & 5.0996 & 84.4467123333333\tabularnewline
Bermuda & 6.8405 & 82.4220155263158\tabularnewline
Canada & 6.42964285714286 & 25.8181562337662\tabularnewline
Finland & 10.2918181818182 & 101.773316363636\tabularnewline
Australia & 5.24459459459459 & 29.3712088588589\tabularnewline
Russia & 7.6725 & 36.7835840909091\tabularnewline
Belgium & 10.1144444444444 & 79.0943277777778\tabularnewline
Kong/China & 5.7175 & 43.3300916666667\tabularnewline
Norway & 10.78 & 142.9008\tabularnewline
Australia/ United Kingdom & 11.595 & 78.50045\tabularnewline
Brazil & 6.33866666666667 & 37.5786123809524\tabularnewline
Denmark & 6.349 & 31.3225655555556\tabularnewline
Sweden & 7.66576923076923 & 32.1031773846154\tabularnewline
Hong Kong/China & 2.044 & 5.47681473684211\tabularnewline
United Kingdom/ Australia & 10.01 & NA\tabularnewline
Ireland & 4.765 & 12.0264571428571\tabularnewline
India & 3.86814814814815 & 30.1490464387464\tabularnewline
Panama/ United Kingdom & 5.93 & NA\tabularnewline
Taiwan & 2.75142857142857 & 2.88253025210084\tabularnewline
Singapore & 3.685 & 9.56976\tabularnewline
South Africa & 4.124 & 4.00395428571429\tabularnewline
Austria & 4.1425 & 5.28153571428571\tabularnewline
Mexico & 3.93764705882353 & 6.03346911764706\tabularnewline
Portugal & 3.88428571428571 & 6.74572857142857\tabularnewline
Thailand & 2.51333333333333 & 7.32475\tabularnewline
Malaysia & 1.71625 & 1.205745\tabularnewline
Turkey & 4.71333333333333 & 7.90682424242424\tabularnewline
Greece & 2.52833333333333 & 1.9028696969697\tabularnewline
Luxembourg & 14.185 & 268.42445\tabularnewline
Islands & 6.67 & NA\tabularnewline
Liberia & 3.78 & NA\tabularnewline
Indonesia & 2.45 & 0.703266666666667\tabularnewline
Israel & 2.06 & 1.66988571428571\tabularnewline
United Kingdom/ Netherlands & 7.54 & NA\tabularnewline
New Zealand & 2.64 & NA\tabularnewline
Africa & 6.82 & 0.1568\tabularnewline
Chile & 1.6025 & 1.49915833333333\tabularnewline
Jordan & 1.33 & NA\tabularnewline
Hungary & 3.37 & 6.4082\tabularnewline
Korea & 15.005 & 196.968966666667\tabularnewline
Poland & 4.41 & NA\tabularnewline
Cayman Islands & 1.66 & 0.71\tabularnewline
Czech Republic & 1.805 & 0.00405000000000001\tabularnewline
Pakistan & 1.23 & NA\tabularnewline
United Kingdom/ South Africa & 2.06 & NA\tabularnewline
France/ United Kingdom & 1.01 & NA\tabularnewline
Philippines & 1.565 & 1.94045\tabularnewline
Bahamas & 1.35 & NA\tabularnewline
Venezuela & 0.98 & NA\tabularnewline
Peru & 0.17 & NA\tabularnewline
\bottomrule
\end{longtable}

\hypertarget{utilize-o-boxplot-para-a-comparauxe7uxe3o-da-questuxe3o-06}{%
\subsection{7- Utilize o boxplot para a comparação da questão
06}\label{utilize-o-boxplot-para-a-comparauxe7uxe3o-da-questuxe3o-06}}

\begin{Shaded}
\begin{Highlighting}[]
\KeywordTok{boxplot}\NormalTok{(df}\OperatorTok{$}\NormalTok{valor[}\KeywordTok{which}\NormalTok{(df}\OperatorTok{$}\NormalTok{pais }\OperatorTok{==}\StringTok{ 'Brazil'}\NormalTok{)],}\DataTypeTok{main =} \StringTok{"Brasil - Valor Empresas"}\NormalTok{)}
\end{Highlighting}
\end{Shaded}

\includegraphics{Pedro-Alves---Trab_01_2020_files/figure-latex/unnamed-chunk-7-1.pdf}

\begin{Shaded}
\begin{Highlighting}[]
\KeywordTok{boxplot}\NormalTok{(df}\OperatorTok{$}\NormalTok{valor[}\KeywordTok{which}\NormalTok{(df}\OperatorTok{$}\NormalTok{pais }\OperatorTok{!=}\StringTok{ 'Brazil'}\NormalTok{)],}\DataTypeTok{main =} \StringTok{"Restante do Mundo - Valor Empresas"}\NormalTok{)}
\end{Highlighting}
\end{Shaded}

\includegraphics{Pedro-Alves---Trab_01_2020_files/figure-latex/unnamed-chunk-7-2.pdf}

\begin{Shaded}
\begin{Highlighting}[]
\KeywordTok{boxplot}\NormalTok{(df}\OperatorTok{$}\NormalTok{vendas[}\KeywordTok{which}\NormalTok{(df}\OperatorTok{$}\NormalTok{pais }\OperatorTok{==}\StringTok{"Brazil"}\NormalTok{)],df}\OperatorTok{$}\NormalTok{vendas[}\KeywordTok{which}\NormalTok{(df}\OperatorTok{$}\NormalTok{pais }\OperatorTok{!=}\StringTok{"Brazil"}\NormalTok{)],}\DataTypeTok{main=} \StringTok{"Vendas"}\NormalTok{,}\DataTypeTok{names =} \KeywordTok{c}\NormalTok{(}\StringTok{"Brasil"}\NormalTok{,}\StringTok{"Restante do Mundo"}\NormalTok{),}\DataTypeTok{xlab=}\StringTok{"US$ Bi"}\NormalTok{,}\DataTypeTok{outline =}\OtherTok{FALSE}\NormalTok{,}\DataTypeTok{horizontal=}\OtherTok{TRUE}\NormalTok{)}
\end{Highlighting}
\end{Shaded}

\includegraphics{Pedro-Alves---Trab_01_2020_files/figure-latex/unnamed-chunk-7-3.pdf}

\end{document}
